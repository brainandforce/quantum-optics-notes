\section{The two-level atom}

The two-level atom is a simple model system that can be used to explore the interactions produced by
electromagnetic waves. As before, we can use a perturbative approach to explore this system. The
Hamiltonian of the atom, $\hat{H}_0$, admits two energy eigenstates, $\phi_1$ and $\phi_2$, with
corresponding energies $E_1$ and $E_2$. If required, we will assume $E_1 < E_2$. In Dirac notation,
we can label these states $\vert 1 \rangle$ and $\vert 2 \rangle$. The energies of these states
differ by $\Delta E = \hbar\Delta\omega$. The perturbation Hamiltonian $\hat{H}_I$ originates from
the electric dipole operator, which is the position operator scaled by the charge:
$\hat{d} = q\hat{x}$. The energy of a dipole is given by the dot product $-\hat{d} \cdot \vec{E}$.
The electric field generated by an electromagnetic wave can be expressed in terms of a cosine
function scaled by the magnitude of the oscillation and the direction of polarization:
\begin{equation}
\hat{H} = \hat{H}_0 - \hat{d} \cdot \vec{\varepsilon} E_0 \cos \omega t
\end{equation}

As we have seen already, the evolution of a wavefunction of the two-level atom perturbed by an
electromagnetic wave can be expressed in terms of time-dependent coefficients that contribute to the
expected time evolution of each eigenstate:
\begin{equation}
\psi\left(\vec{x},t\right) = 
    c_1\left(t\right) e^{-\frac{i}{\hbar} E_1 t} \phi_1 +
    c_2\left(t\right) e^{-\frac{i}{\hbar} E_2 t} \phi_2
\end{equation}

From the previous derivation for the time evolution of the coefficients, we can find expressions
for the evolution of $c_1\left(t\right)$ and $c_2\left(t\right)$:
\begin{flalign}
\partialderivative{t} c_1\left(t\right)
& =
\frac{1}{i\hbar}\sum_{n = 1}^{2} c_n\left(t\right) e^{-\frac{i}{\hbar}\left(E_n - E_1\right) t}
    \langle \phi_1 \vert -\hat{d} \cdot \vec{\varepsilon} E_0 \cos \omega t \vert \phi_n \rangle
\\
\partialderivative{t} c_2\left(t\right)
& =
\frac{1}{i\hbar} \sum_{n = 1}^{2} c_n\left(t\right) e^{-\frac{i}{\hbar}\left(E_n - E_2\right) t}
    \langle \phi_2 \vert -\hat{d} \cdot \vec{\varepsilon} E_0 \cos \omega t \vert \phi_n \rangle
\end{flalign}
To shorten this expression, we define the \textit{dipole matrix}, a matrix representation of the
dipole operator in terms of our eigenbasis and the electric field. The dot product with the
polarization vector $\vec{\varepsilon}$ may be pulled out of the expression. 
\begin{flalign}
d^{\varepsilon}_{ab}
& =
\langle \phi_a \vert -\hat{d} \cdot \vec{\varepsilon} \, \vert \phi_b \rangle
\\
& = -\vec{\varepsilon} \cdot \langle \phi_a \vert \hat{d} \vert \phi_b \rangle
\\
& = -q \vec{\varepsilon} \cdot \langle \phi_a \vert \hat{x} \vert \phi_b \rangle
\end{flalign}
Because the position operator $\hat{x}$ is Hermitian, the dipole moment operator is also Hermitian,
corresponding to the fact that the dipole moment is an observable quantity. This also means that
$d^{\varepsilon}_{ab} = d^{\varepsilon *}_{ba}$.

Now we can rephrase our equation in terms of a matrix and vector in the eigenbasis given by the
unperturbed Hamiltonian $\hat{H}_0$. This allows us to write the unperturbed Hamiltonian as a
diagonal matrix with the entries being the energies of each state. The Schrödinger equation then
looks like:
\begin{equation}
i\hbar\partialderivative{t} 
\begin{bmatrix}
    c_1\left(t\right) e^{-\frac{i}{\hbar} E_1 t} \\
    c_2\left(t\right) e^{-\frac{i}{\hbar} E_2 t}
\end{bmatrix}
=
\left(
\begin{bmatrix} E_1 & 0 \\ 0 & E_2 \end{bmatrix} + E_0 \cos \left(\omega t\right)
\begin{bmatrix}
    d^{\varepsilon}_{11} & d^{\varepsilon}_{12} \\
    d^{\varepsilon}_{21} & d^{\varepsilon}_{22}
\end{bmatrix}
\right) 
\begin{bmatrix}
    c_1\left(t\right) e^{-\frac{i}{\hbar} E_1 t} \\
    c_2\left(t\right) e^{-\frac{i}{\hbar} E_2 t}
\end{bmatrix}
\end{equation}
However, we can absorb the diagonal components of the perturbation matrix into the energies of the
eigenstates, making the substitutions $E_1 \to E_1 + d^{\varepsilon}_{11}$ and 
$E_2 \to E_2 + d^{\varepsilon}_{22}$. We can then define $D = d^{\varepsilon}_{12}$, and use the
Hermitian property to simplify even further:
\begin{equation}
i\hbar\partialderivative{t} 
\begin{bmatrix}
    c_1\left(t\right) e^{-\frac{i}{\hbar} E_1 t} \\
    c_2\left(t\right) e^{-\frac{i}{\hbar} E_2 t}
\end{bmatrix}
=
\left(
\begin{bmatrix} E_1 & 0 \\ 0 & E_2 \end{bmatrix} + E_0 \cos \left(\omega t\right)
\begin{bmatrix} 0 & D \\ D^* & 0 \end{bmatrix}
\right) 
\begin{bmatrix}
    c_1\left(t\right) e^{-\frac{i}{\hbar} E_1 t} \\
    c_2\left(t\right) e^{-\frac{i}{\hbar} E_2 t}
\end{bmatrix}
\end{equation}
We can substitute this simplification back into our original equations, exploiting the fact that our
perturbation is zero along the diagonal. This means that the time evolution of one coefficient
depends only on the other coefficient:
\begin{flalign}
\partialderivative{t} c_1\left(t\right)
& =
\frac{1}{i\hbar} c_2\left(t\right) e^{-\frac{i}{\hbar}\left(E_2 - E_1\right) t}
    E_0 \cos \left(\omega t\right) D
\\
\partialderivative{t} c_2\left(t\right)
& =
\frac{1}{i\hbar} c_1\left(t\right) e^{-\frac{i}{\hbar}\left(E_1 - E_2\right) t}
    E_0 \cos \left(\omega t\right) D^*
\end{flalign}
We can perform one more simplification with the identification of $\Delta E = E_2 - E_1$ and using
a frequency representation instead, $\Delta \omega$:
\begin{flalign}
\partialderivative{t} c_1\left(t\right)
& =
\frac{D^* E_0 }{i\hbar} \cos \left(\omega t\right) e^{-i \Delta \omega t} c_2\left(t\right)
\\
\partialderivative{t} c_2\left(t\right)
& =
\frac{D E_0}{i\hbar}    \cos \left(\omega t\right) e^{i \Delta \omega t} c_1\left(t\right)
\end{flalign}
If we assume that $D$ is real, the quantity $\frac{D E_0}{\hbar}$ is the \textit{Rabi frequency},
which we represent with $\Omega_0$. We can also describe the cosines in terms of imaginary
exponentials:
\begin{flalign}
\partialderivative{t} c_1\left(t\right)
& =
-i\Omega_0 \frac{e^{i\omega t} + e^{-i\omega t}}{2} e^{-i \Delta \omega t} c_2\left(t\right)
\\
\partialderivative{t} c_2\left(t\right)
& =
-i\Omega_0 \frac{e^{i\omega t} + e^{-i\omega t}}{2} e^{i \Delta \omega t} c_1\left(t\right)
\end{flalign}
By combining the exponential terms, we get
\begin{flalign}
\partialderivative{t} c_1\left(t\right)
& =
-\frac{i\Omega_0}{2} 
\left(
    e^{i\left(\omega - \Delta \omega\right) t} + 
    e^{-i\left(\omega + \Delta \omega\right) t}\right)
c_2\left(t\right)
\\
\partialderivative{t} c_2\left(t\right)
& =
-\frac{i\Omega_0}{2} 
\left(
    e^{-i\left(\omega - \Delta \omega\right) t} + 
    e^{i\left(\omega + \Delta \omega\right) t}\right)
c_1\left(t\right)
\end{flalign}
The term $\omega - \Delta \omega$ is the detuning that we identified earlier. the detuning parameter
corresponds to a slow oscillation, but the other parameter $\omega + \Delta \omega$ corresponds to a
very fast oscillation that can be averaged out. Therefore, we can neglect the exponential term that
contains the sum of frequencies when our detuning parameter is small: this is the \textit{rotating
wave approximation}.
\begin{flalign}
\partialderivative{t} c_1\left(t\right)
& =
-\frac{i\Omega_0}{2} e^{i\left(\omega - \Delta \omega\right) t} c_2\left(t\right)
\\
\partialderivative{t} c_2\left(t\right)
& =
-\frac{i\Omega_0}{2} e^{-i\left(\omega - \Delta \omega\right) t}c_1\left(t\right)
\end{flalign}
If the detuning parameter is large, we cannot ignore this parameter, so the approximation becomes
invalid.

\subsection{The Bloch sphere}

The wavefunction of a two-level atom can be written in terms of its eigenstates:
\begin{equation}
\psi\left(\vec{x},t\right) =
    c_1 \phi_1\left(\vec{x},t\right) + c_2 \phi_2\left(\vec{x},t\right)
\end{equation}
Because we know that $c_1$ and $c_2$ are complex, this gives our system four degrees of freedom --
at least, to start. However, we know that wavefunctions are identical up to multiplication by a
complex constant, which removes two degrees of freedom from the system.

It would be convenient to come up with alternate description of the state space of the two-level
atom that did not include the redundancies associated with complex coefficients. We can come up with
such a formulation by expressing the normalization condition for the wavefunction:
\begin{equation}
\int \left(c_1 \phi_1\left(\vec{x},t\right) + c_2 \phi_2\left(\vec{x},t\right)\right)^\dagger
    \left(c_1 \phi_1\left(\vec{x},t\right) + c_2 \phi_2\left(\vec{x},t\right)\right) d \vec{x}
    = 1
\end{equation}
We know that the eigenstates form an orthonormal basis, so we can simplify this drastically:
\begin{equation}
c_1^\dagger c_1 + c_2^\dagger c_2 = 1
\end{equation}
To split this into two factors, we can express $c_1$ and $c_2$ in terms of two angular parameters, 
$\theta$ and $\varphi$, which satisfy the relation:
\begin{equation}
\left(\cos\theta\right)^2 + e^{-i\varphi} e^{i\varphi} \left(\sin\theta\right)^2  = 1
\end{equation}
and thus gives us a new expression for the wavefunction involving only two real parameters:
\begin{equation}
\psi = \phi_1 \cos\theta + \phi_2 e^{i\varphi} \sin\theta
\end{equation}
There is a convenient geometric interpretation of the angular parameters: they represent coordinates
on the surface of a sphere. This representation is the \textit{Bloch sphere}. Its surface represents
all possible pure states of the system. Mixed states, which will come up later, correspond to points
in the interior of the sphere.

\subsection{The density operator}

Consider a two-level atom that has been prepared in some state $\psi\left(\vec{x},0\right)$. We know
it was prepared in one of two states, $\phi_1\left(\vec{x},0\right)$ or
$\phi_2\left(\vec{x},0\right)$, with equal probability. How can we describe this state? Our first
instinct may be to describe the state as a superposition of the two states:
\begin{equation}
\psi = \frac{1}{\sqrt{2}}\left(\phi_1 + \phi_2\right)
\end{equation}
But this raises a problem: this tells us that the system was prepared in exactly the superposition
of states, not in a probabilistic mixture of the two pure states. An experimentalist will not know
with certainty that a system is prepared in exactly in one state over the other, but superpositions
do not convey this ideay. The uncertainty in state preparation requires its own treatment separate
from the quantum mechanical probability amplitudes.

The solution to this problem is to define the density operator $\hat{\rho}$. Our knowledge of the
state of the system can be described by this operator instead of a wavefunction. However, we use
wavefunctions to construct the operator. In the system we want to model, we know that the system is
certainly in either $\phi_1$ or $\phi_2$, but certainly not in any superposition state.

A \textit{pure state} is a state that can be represented as the product of a state vector with
itself. For instance, if we managed to generate exactly the state $c_1\phi_1 + c_2\phi_2$, we could
describe it in terms of the eigenbasis:
\begin{equation}
\hat{\rho}_{ab} =
\begin{bmatrix}
    c_1^* c_1   &   c_1^* c_2   \\
    c_2^* c_1   &   c_2^* c_2   
\end{bmatrix}
\end{equation}
