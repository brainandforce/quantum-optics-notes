\section*{Prerequisites}

These notes assume familiarity with the mathematics of introductory quantum mechanics as taught in
an undergraduate chemistry course.

\subsection*{Wavefunctions}

Every quantum state $\psi$ can be thought of as a normalized square-integrable function,
$\psi\left(\vec{x},t\right)$, unique up to multiplication by a nonzero complex scalar. Because
wavefunctions may be freely added to each other or scaled, they form a vector space -- specifically,
an infinite-dimensional complex projective Hilbert space. A \textit{Hilbert space} is a vector space
that admits some notion of distance, which we can obtain from the \textit{overlap integral}:
\begin{equation*}
\int \psi_a\left(\vec{x},t\right)^* \psi_b\left(\vec{x},t\right) d\vec{x}
\end{equation*}
This integral serves the same purpose as the dot product, and describes the degree of similarity
between $\psi_a$ and $\psi_b$. The star represents complex conjugation, critical for obtaining
real-valued dot products from complex-valuedfunctions.

In many cases, we will be interested in a restricted subset of functions in this space. We can
express a state $\psi$ in terms of basis functions $\phi$:
\begin{equation*}
\psi = \sum_{n} c_n \phi_n
\end{equation*}
The choice of basis is arbitrary, as long as all $\psi_n$ are linearly independent. Now that the
basis is known, we can describe $\phi$ as a column vector:
\begin{equation*}
\psi = \begin{bmatrix}
c_1 \\ c_2 \\ c_3 \\ ...
\end{bmatrix}
\end{equation*}
In a vector representation, the complex conjugate is insufficient to calculate the overlap integral.
Instead, we use the \textit{adjoint}, which combines complex conjugation with transposition:
\begin{equation*}
\psi^\dagger = \begin{bmatrix}
c_1^* & c_2^* & c_3^* & ...
\end{bmatrix}
\end{equation*}
This allows us to represent the dot product as a matrix multiplication.

The final form of notation we will use is the Dirac bra-ket notation. A wavefunction is described by
a \textit{ket}, $\vert\psi\rangle$, and its adjointis the corresponding \textit{bra},
$\langle\psi\vert$. The dot product of two wavefunctions is neatly described with this notation:
\begin{equation*}
\langle \psi_a \vert \psi_b \rangle = 
\int \psi_a\left(\vec{x},t\right)^* \psi_b\left(\vec{x},t\right) d\vec{x}
\end{equation*}
The \textit{bra-ket} expression implicitly includes the integral.

\subsection*{Operators}

If functions can be thought of as vectors, operators are matrices which act on those vectors. By
convention, an operator will be given a hat : for example, the momentum operator is
$\hat{p} = -i\hbar\frac{d}{dx}$.

Operators corresponding to observable quantities are Hermitian, meaning they are equal to their
adjoint. However, not all operators are Hermitian: for instance, the propagator, as well as the
creation and annihilation operators. For an operator $\hat{A}$ which corresponds to an observable
quantity $A$, we can calculate its \textit{expectation value}, the approximate value of that
observable for a wavefunction $\psi$, with an integral:
\begin{equation*}
\langle \hat{A} \rangle = \langle \psi \vert \hat{A} \vert \psi \rangle = 
\int \psi\left(\vec{x},t\right)^* \hat{A} \psi\left(\vec{x},t\right) d\vec{x}
\end{equation*}
