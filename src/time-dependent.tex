\section{Perturbation theory of quantum state evolving in time}

\subsection{Time evolution of quantum states}

A quantum system can be described in terms of a wavefunction $\psi$ that contains all of the
observable information associated with the system. These observables are encoded by Hermitian linear
operators that act on $\psi$. Of particular note is the Hamiltonian operator $\hat{H}$, which
describes the time evolution of $\psi$. This defines the \textit{time-dependent Schrödinger
equation}:
\begin{equation}
i \hbar \frac{\partial}{\partial t} \psi\left(x,t\right) = \hat{H}\psi\left(x,t\right) 
\end{equation}
The wavefunction $\psi$ can be described as a normalized square-integrable function. If we describe
$\psi$ as a function from a spatial coordinate and time to a complex scalar, multiplying $\psi$ by
a unit complex number (phase factor) does not affect the values of observable quantities associated
with $\psi$, so they describe the same state.

To describe the evolution of $\psi\left(\vec{x},t\right)$, we define the \textit{propagator} as an
exponentiation of the Hamiltonian which acts upon the state at a known time $t$ to generate the
state at a different time $t + \Delta t$:
\begin{equation}
\psi\left(\vec{x},t + \Delta t\right) = \psi\left(\vec{x},t\right) e^\frac{-i\hat{H}\Delta t}{\hbar}
\end{equation}
We can derive this by describing the wavefunction at a $t + \Delta t$ as a Taylor expansion of
$\psi\left(\vec{x},t\right)$:
\begin{flalign*}
\psi\left(\vec{x},t + \Delta t\right) & = \sum_{n = 0}^{\infty} \left[
    \frac{\Delta t^n}{n!} \frac{\partial^n}{\partial t^n} \psi\left(\vec{x},t\right)
    \right]
\end{flalign*}
Now we can use the time-dependent Schrödinger equation to substitute the wavefunction derivatives in
terms of applying the Hamiltonian operator to the wavefunction, starting by introducing factors of
$i \hbar$ to the derivative:
\begin{flalign*}
\psi\left(\vec{x},t + \Delta t\right) & = \sum_{n = 0}^{\infty} \left[
    \frac{1}{n!} \left(\frac{\Delta t}{i\hbar}\right)^n
    \left(i \hbar\right)^n \frac{\partial^n}{\partial t^n} \psi\left(\vec{x},t\right)\right]    \\
& = \sum_{n = 0}^{\infty} \left[
    \frac{1}{n!} \left(\frac{\Delta t}{i\hbar}\right)^n
    \left(i \hbar \frac{\partial}{\partial t}\right)^n \psi\left(\vec{x},t\right)\right]    \\
& = \sum_{n = 0}^{\infty} \left[ 
    \frac{1}{n!} \left(\frac{\Delta t}{i\hbar}\right)^n
    \hat{H}^n \psi\left(\vec{x},t\right)\right] \\
& = \sum_{n = 0}^{\infty} \left[ 
    \frac{1}{n!} \left(\frac{\hat{H}\Delta t}{i\hbar}\right)^n \psi\left(\vec{x},t\right)\right]  \\
& = \sum_{n = 0}^{\infty} \left[ 
    \frac{1}{n!} \left(-\frac{i\hat{H}\Delta t}{\hbar}\right)^n \psi\left(\vec{x},t\right)\right] \\
\end{flalign*}
Because the Hamiltonian is a linear operator, we can use its additive property to pull the
$\psi\left(\vec{x},t\right)$ terms out of the sum:
\begin{flalign*}
\sum_{n = 0}^{\infty} \left[ 
    \frac{1}{n!} \left(-\frac{i\hat{H}\Delta t}{\hbar}\right)^n \psi\left(\vec{x},t\right)\right]
& = \sum_{n = 0}^{\infty} \left[ 
    \frac{1}{n!} \left(-\frac{i\hat{H}\Delta t}{\hbar}\right)^n\right] \psi\left(\vec{x},t\right)
\end{flalign*}
What we find is that the sum of operators is actually an exponentiation of
$-\frac{i\hat{H}\Delta t}{\hbar}$, using the definiton of the Taylor series of the exponential:
\begin{flalign*}
\sum_{n = 0}^{\infty} \left[ 
    \frac{1}{n!} \left(-\frac{i\hat{H}\Delta t}{\hbar}\right)^n\right] \psi\left(\vec{x},t\right)
& = e^{-\frac{i\hat{H}\Delta t}{\hbar}}\psi\left(\vec{x},t\right)
\end{flalign*}
The new operator we derive is the \textit{propagator}. Any state subject to a Hamiltonian will
evolve in time according to the action of the propagator upon the wavefunction.

Some Hamiltonians admit \textit{stationary states} whose observables do not change over time. The
time evolution operator applied to a stationary state $\varphi\left(\vec{x},t\right)$ will be equal
to multiplication by a unit complex scalar, generated by exponentiating a real scalar:
\begin{equation}
\varphi\left(\vec{x},t + \Delta t\right) = e^{-\frac{i\hat{H}t}{\hbar}}\varphi\left(\vec{x},t\right)
= e^{-\frac{iEt}{\hbar}} \varphi\left(\vec{x},t\right)
\end{equation}
The real scalar $E = \hbar \omega$ is the energy of the system. This allows us to write an equation
that relates the Hamiltonian operator and the energy scalar for the stationary states. This equation
is the \textit{time-independent Schrödinger equation}:
\begin{equation}
\hat{H}\varphi\left(\vec{x},t\right) = E\varphi\left(\vec{x},t\right)
\end{equation}
Because this is an eigenvalue equation, the stationary states of $\hat{H}$ are also known as
\textit{eigenstates}, which correspond to eigenvectors of $\hat{H}$, with the eigenvalues being the
energies of the states.

If the Hamiltonian contains an infinitely deep potential well and an infinite number of eigenstates,
any wavefunction $\psi$ can be expressed as a linear combination of eigenstates:
\begin{equation}
\psi\left(\vec{x},t\right) = \sum_{n = 0}^{\infty} c_n \varphi_n\left(\vec{x},t\right)
\end{equation}
The time evolution of a linear combination of eigenstates can be calculated by propagating each
component of the eigenbasis using the associated energies or frequencies of each state:
\begin{equation}
\psi(\vec{x},t + \Delta t)
    = \sum_{n = 0}^{\infty} e^{-\frac{i E_n \Delta t}{\hbar}} c_n \varphi_n\left(\vec{x},t\right) 
    = \sum_{n = 0}^{\infty} e^{-i\omega_n \Delta t} c_n \varphi_n\left(\vec{x},t\right) 
\end{equation}
Note that the eigenstates all undergo rotations at their own frequencies $\omega_n$. Although a
global shift of phase induced by multiplying $\psi$ by a complex scalar does not change the
observable state, shifting the phases of each individual component by different amounts 
\textit{does} change the state over time.

One last point to remember is that Hamiltonian operators $\hat{H}$ may be shifted by some constant
energy $E_0$, and that value does not affect the dynamics of the system:
\begin{flalign*}
\psi(\vec{x},t + \Delta t)
& = e^{-\frac{i\left(\hat{H} + E_0\right)\Delta t}{\hbar}} \psi(\vec{x},t)  \\
& = e^{-\frac{i\hat{H}\Delta t}{\hbar}}e^{-\frac{i E_0 \Delta t}{\hbar}} \psi(\vec{x},t)
\end{flalign*}
Because $E_0$ is a constant, the second exponential is a complex constant, and it only serves to
change the rate of phase rotation of the wavefunction, which has no observable consequences.
However, shifting the Hamiltonian so that the energy of an eigenstate of interest is zero allows for
the propagation of that state to be described with an identity operator:
\begin{flalign*}
\varphi_n(\vec{x},t + \Delta t)
& = e^{-\frac{i\left(\hat{H} - E_n\right)\Delta t}{\hbar}} \varphi_n(\vec{x},t) \\
& = e^{-\frac{i\hat{H}\Delta t}{\hbar}}e^{\frac{i E_n \Delta t}{\hbar}} \varphi_n(\vec{x},t)   \\
& = e^{-\frac{i E_n \Delta t}{\hbar}}e^{\frac{i E_n \Delta t}{\hbar}} \varphi_n(\vec{x},t)  \\
& = \varphi_n(\vec{x},t)
\end{flalign*}
This allows us to describe the relative differences in energy or frequency between a reference
eigenstate and another eigenstate of interest. Only the relative differences in frequencies between
states have physically observable consequences.

\subsection{The Heisenberg picture}

The time-dependent Schrödinger equation describes the evolution of a quantum state over time.
Although the observables associatated with the state are fixed, the operators that act upon the
state to generate the observable are not. But this choice is arbitrary: it is entirely possible to
develop a quantum theory of fixed states and operators which propagate over time to generate the
changes in observables. This formulation is known as the \textit{Heisenberg picture}.

To derive this picture, we start with the calculation of expectation values from quantum state
$\psi\left(\vec{x},t\right)$. For some observable $\hat{O}$, the expectation value is
\begin{equation}
\langle\hat{O}\rangle\left(t\right) = \int 
    \psi\left(\vec{x},t\right)^\dagger \hat{O} \psi\left(\vec{x},t\right) d\vec{x}
\end{equation}
But we can describe the wavefunctions at time $t$ in terms of $\psi\left(\vec{x},0\right)$ 
(shortened to $\psi\left(\vec{x}\right)$) and the propagator:
\begin{equation}
\langle\hat{O}\rangle\left(t\right) = \int
    \left(e^{-\frac{i\hat{H}t}{\hbar}}\psi\left(\vec{x}\right)\right)^\dagger 
    \hat{O} e^{-\frac{i\hat{H}t}{\hbar}} \psi\left(\vec{x}\right) d\vec{x}
\end{equation}
When we apply the adjoint to the application of an operator to a wavefunction, we get the reverse
product of the adjoints of the operator and the wavefunction:
\begin{equation}
\langle\hat{O}\rangle\left(t\right) = \int
    \psi\left(\vec{x}\right)^\dagger \left(e^{-\frac{i\hat{H}t}{\hbar}}\right)^\dagger
    \hat{O} e^{-\frac{i\hat{H}t}{\hbar}} \psi\left(\vec{x}\right) d\vec{x}
\end{equation}
We've managed to transfer the action of the propagator from the wavefunction to the operator. In
general, operators evolve according to the following equation:
\begin{equation}
\hat{O}\left(t\right) =
\left(e^{-\frac{i\hat{H}t}{\hbar}}\right)^\dagger \hat{O} e^{-\frac{i\hat{H}t}{\hbar}}
\end{equation}
The propagator, being the exponential of a Hermitian operator, is \textit{unitary}, meaning that it
preserves dot products. This also means its adjoint is equal to its inverse, so we can rewrite the
equation:
\begin{equation}
\hat{O}\left(t\right) = e^{\frac{i\hat{H}t}{\hbar}} \hat{O} e^{-\frac{i\hat{H}t}{\hbar}}
\end{equation}

In many cases, we are interested in the changes in observable quantities, such as position or
momentum, over time. What happens if we take derivatives of the operators with respect to time?
\begin{flalign*}
\frac{d}{dt} \hat{O}\left(t\right)
& = \frac{d}{dt}\left(e^{\frac{i\hat{H}t}{\hbar}} \hat{O} e^{-\frac{i\hat{H}t}{\hbar}}\right)   \\
& = \frac{d}{dt}\left(e^{\frac{i\hat{H}t}{\hbar}}\right) \hat{O} e^{-\frac{i\hat{H}t}{\hbar}} +
    e^{\frac{i\hat{H}t}{\hbar}} \hat{O} \frac{d}{dt}\left(e^{-\frac{i\hat{H}t}{\hbar}}\right)   \\
& = \frac{i}{\hbar} \left(
    e^{\frac{i\hat{H}t}{\hbar}} \hat{H}^\dagger \hat{O} e^{-\frac{i\hat{H}t}{\hbar}} -
    e^{\frac{i\hat{H}t}{\hbar}} \hat{O} \hat{H} e^{-\frac{i\hat{H}t}{\hbar}}\right)
\end{flalign*}
We use the fact that $\hat{H}$ is a Hermitian operator, meaning that it is equal to its adjoint:
\begin{flalign*}
\frac{d}{dt} \hat{O}\left(t\right)
& = \frac{i}{\hbar} \left(
    e^{\frac{i\hat{H}t}{\hbar}} \hat{H} \hat{O} e^{-\frac{i\hat{H}t}{\hbar}} -
    e^{\frac{i\hat{H}t}{\hbar}} \hat{O} \hat{H} e^{-\frac{i\hat{H}t}{\hbar}}\right) \\
& = \frac{i}{\hbar} \left(
    e^{\frac{i\hat{H}t}{\hbar}} \hat{H} e^{-\frac{i\hat{H}t}{\hbar}} 
    e^{\frac{i\hat{H}t}{\hbar}} \hat{O} e^{-\frac{i\hat{H}t}{\hbar}} -
    e^{\frac{i\hat{H}t}{\hbar}} \hat{O} e^{-\frac{i\hat{H}t}{\hbar}}
    e^{\frac{i\hat{H}t}{\hbar}} \hat{H} e^{-\frac{i\hat{H}t}{\hbar}}\right)    \\
& = \frac{i}{\hbar} \left(\hat{H}\left(t\right) \hat{O} \left(t\right) -
    \hat{O} \left(t\right) \hat{H}\left(t\right)\right)
\end{flalign*}
It should now be apparent that the quantity in parentheses is the commutator of $\hat{H}$ and
$\hat{O}$:
\begin{equation}
\frac{d}{dt} \hat{O}\left(t\right)
    = \frac{i}{\hbar} \left[\hat{H}\left(t\right), \hat{O}\left(t\right)\right]
    = \frac{1}{i\hbar}\left[\hat{O}\left(t\right), \hat{H}\left(t\right)\right]
\end{equation}
This is the \textit{Heisenberg equation}. Note that the Hamiltonian itself may be propagated over
time: this will be important for the next portion.

\subsection{The interaction picture}

As hinted by the derivation of the Heisenberg picture, there are not just two different perspectives
of a quantum system, but an infinite number of possible perspectives that depend on the choice of
unitary operator. This is useful in perturbation theory, where we break up the Hamiltonian into an
analytically solvable term and a small perturbation.

Consider a Hamiltonian that consists of two terms: a base term $\hat{H}_0$, and an interaction term
$\hat{H}_I$, which in the cases we consider in the future may be time dependent. The associated
propagator can be naturally divided:
\begin{flalign}
\hat{U} & = e^{-\frac{i}{\hbar}\left(\hat{H}_0 + \hat{H}_I\right)t}       \\
    & = e^{-\frac{i}{\hbar}\hat{H}_0 t} e^{-\frac{i}{\hbar}\hat{H}_I t}   \\
    & = \hat{U}_0 \hat{U}_I = \hat{U}_I \hat{U}_0
\end{flalign}
We can assign each component of the propagator to different duties: $\hat{U}_0$ transforms the
operators and $\hat{U}_I$ transforms the states:
\begin{flalign}
\psi\left(\vec{x},t\right) & = e^{-\frac{i}{\hbar}\hat{H}_I t} \psi\left(\vec{x},0\right)   \\
\hat{O}(t) & = e^{\frac{i}{\hbar}\hat{H}_0 t} \hat{O}(0) e^{-\frac{i}{\hbar}\hat{H}_0 t}
\end{flalign}
This gives us the equations of motion for both the states and operators:
\begin{flalign}
i\hbar \frac{d}{dt} \psi\left(\vec{x},t\right) & = \hat{H}_I \psi\left(\vec{x},t\right) \\
\frac{d}{dt}\hat{O}(t) & = \frac{1}{i\hbar}\left[\hat{O}(t), \hat{H}_0(t)\right]
\end{flalign}
Because $\hat{H}_0$ is constant over time, we have a simpler expression for the time propagation
of an operator, and the propagation of the wavefunction is given only in terms of the interaction
Hamiltonian $\hat{H}_I$. We could also reverse this to obtain a complementary picture. This division
of the propagator generates the \textit{interaction picture}.

\subsection{Incorporating perturbation theory}

The generic Hamiltonian we discussed previously is amenable to treatment with perturbation theory.
The idea behind perturbation theory is that we can describe a Hamiltonian $\hat{H}$ that would be
difficult or impossible to solve analytically as the sum of an analytically solvable Hamiltonian
$\hat{H}_0$ and a small perturbation $\hat{H}_I$. The solutions to $\hat{H}$ will resemble those Of
$\hat{H}_0$, but with correction terms that can be calculated by $\hat{H}_I$. Provided that
$\hat{H}_I$ is sufficiently small, the corrections can be written in terms of a converging power
series.

Consider an atom with Hamiltonian $\hat{H}_0$ subject to a perturbation $\hat{H}_I$ -- in the
context of quantum optics, $\hat{H}_I$ can be thought of as an electromagnetic wave. At some time
$t$, the wavefunction $\psi\left(\vec{x},t\right)$ can be represented in terms of the eigenstates of
$\hat{H}_0$, $\phi^{(0)}_n$:
\begin{equation}
\psi\left(\vec{x},t\right) = \sum_n c_n\left(t\right) e^{-\frac{i}{\hbar}E_n t} \phi^ {(0)}_n
\end{equation}
Although we used the eigenstates of $\hat{H}_0$ as a basis for the wavefunction, its time evolution
is subject to the full Hamiltonian, and therefore, the coefficients $c_n$ are not fixed, but vary
over time in a way that depends on the perturbation. To understand exactly how they vary, we can
insert the wavefunction into the time-dependent Schrödinger equation:
\begin{equation}
i\hbar\frac{\partial}{\partial t} \sum_n\left[
    c_n\left(t\right) e^{-\frac{i}{\hbar}E_n t} \phi^{(0)}_n
\right] = \left(\hat{H}_0 + \hat{H}_I\right) \sum_n \left[
    c_n\left(t\right) e^{-\frac{i}{\hbar}E_n t} \phi^{(0)}_n
\right]
\end{equation}
We'll use the product rule to split the derivative on the left into portions that depend on the
coefficients $c_n$ and the propagated eigenstate of $\hat{H}_0$. Linearity allows us to omit the
summation for these steps:
\begin{equation}
i\hbar \frac{\partial}{\partial t}
\left[c_n\left(t\right) e^{-\frac{i}{\hbar}E_n t} \phi^{(0)}_n\right]
= i\hbar \frac{\partial c_n}{\partial t} e^{-\frac{i}{\hbar}E_n t} \phi^ {(0)}_n + 
    i\hbar c_n\left(t\right) 
    \frac{\partial}{\partial t}\left[e^{-\frac{i}{\hbar}E_n t} \phi^{(0)}_n\right]
\end{equation}
We'll apply the Hamiltonian in a similar manner. Because $c_n\left(t\right)$ only scales the basis
wavefunctions, it can be pulled out of the Hamiltonian operator.
\begin{flalign}
\hat{H} \left[c_n\left(t\right) e^{-\frac{i}{\hbar}E_n t} \phi^{(0)}_n\right]
& = c_n\left(t\right) \hat{H} \left[e^{-\frac{i}{\hbar}E_n t} \phi^{(0)}_n\right]
\\
& = c_n\left(t\right) \left(\hat{H}_0 + \hat{H}_I\right) 
    \left[e^{-\frac{i}{\hbar}E_n t} \phi^{(0)}_n\right]
\\
& = c_n\left(t\right) \hat{H}_0\left[e^{-\frac{i}{\hbar}E_n t} \phi^{(0)}_n\right] +
    c_n\left(t\right) \hat{H}_I\left[e^{-\frac{i}{\hbar}E_n t} \phi^{(0)}_n\right] 
\end{flalign}
What we find is that we have terms originating from the time-dependent Schrödinger equation of
$\hat{H}_0$:
\begin{equation}
i\hbar c_n\left(t\right) \sum_n
\frac{\partial}{\partial t}\left[e^{-\frac{i}{\hbar}E_n t} \phi^{(0)}_n\right]
= \sum_n
c_n\left(t\right) \hat{H}_0\left[e^{-\frac{i}{\hbar}E_n t} \phi^{(0)}_n\right]
\end{equation}
We can therefore remove these terms from the time-dependent Schrödinger equation of $\hat{H}$ and
only deal with the ones arising from the perturbation:
\begin{equation}
\sum_n i\hbar \frac{\partial c_n}{\partial t} e^{-\frac{i}{\hbar}E_n t} \phi^{(0)}_n 
=
\sum_n c_n\left(t\right) \hat{H}_I\left[e^{-\frac{i}{\hbar}E_n t} \phi^{(0)}_n\right]
\end{equation}
If we want to express the evolution of a single coefficient $c_k$, we can project this equation onto
$\phi^{(0)}_k$. Because the eigenstates form an orthonormal basis, we know that terms where
$k \neq n$ become zero, allowing us to remove a sum on the left:
\begin{flalign}
\sum_n \int 
\phi^{(0)*}_k i\hbar \frac{\partial c_n}{\partial t} e^{-\frac{i}{\hbar}E_n t} \phi^{(0)}_n
d\vec{x}
& = 
\sum_n \int 
c_n\left(t\right) \phi_k^{(0)*} \hat{H}_I\left[e^{-\frac{i}{\hbar}E_n t} \phi^{(0)}_n\right]
d\vec{x}
\\
i\hbar \frac{\partial c_k}{\partial t} e^{-\frac{i}{\hbar}E_k t}
& = 
\sum_n c_n\left(t\right) 
\int \phi_k^{(0)*} \hat{H}_I\left[e^{-\frac{i}{\hbar}E_n t} \phi^{(0)}_n\right] d\vec{x}
\end{flalign}
Because the propagator behaves like a scalar for the eigenstates, we can remove the propagator from
the Hamiltonian operator and combine it with $c_n\left(t\right)$. This allows us to simplify the
left side:
\begin{flalign}
i\hbar \frac{\partial c_k}{\partial t} e^{-\frac{i}{\hbar}E_k t}
& = 
\sum_n c_n\left(t\right) e^{-\frac{i}{\hbar}E_n t}
\int \phi_k^{(0)*} \hat{H}_I \phi^{(0)}_n d\vec{x}
\\
i\hbar \frac{\partial c_k}{\partial t}
& = 
\sum_n c_n\left(t\right) e^{-\frac{i}{\hbar}\left(E_n - E_k\right) t} 
\int \phi_k^{(0)*} \hat{H}_I \phi^{(0)}_n d\vec{x}
\end{flalign}

Consider a real system with an atom whose Hamiltonian is $\hat{H}_0$ and a perturbation introduced
by an electromagnetic wave, contributing the $\hat{H}_I$ term. The wave can cause the electrons to
move from the atomic ground state $\phi_1$ to some other state $\phi_k$. To understand what happens
here, let's treat the perturbation explicitly. An electromagnetic wave can be described by an
expression of the form 
\begin{equation}
\gamma\left(\vec{x},t\right) = \cos \left(\vec{k} \cdot \vec{x} - \omega t + \varphi\right)
\end{equation}
The vector $\vec{k}$ is the wavevector, which represents the direction in which the wave propagates
through space. We know that $\vert\vec{k}\vert$ is proportional to $\omega$ -- the conversion
factor is the speed of light:
\begin{equation}
\omega = c\vert\vec{k}\vert
\end{equation}
Also recall that the energy is proportional to the frequency:
\begin{equation}
E = \hbar \omega
\end{equation}
However, we can make a simplifying assumption: for a real atom interacting with photons in the
visible spectrum, the wavelength is far larger than the atom. Therefore, we can ignore the spatial
component and treat it as a purely temporal wave. As one last simplification, we can remove the
phase factor $\varphi$ and express the wave in terms of a complex exponential:
\begin{equation}
\hat{H}_I\left(t\right) = \hat{H}_I e^{-i\omega t} = \hat{H}_I e^{-\frac{iEt}{\hbar}}
\end{equation}

Now we can apply this to our time evolution expression to understand how the electromagnetic wave
changes the probability of the atom being found in some other state $\phi_k$ starting with the
ground state $\phi^{(0)}_1$ at $t = 0$:
\begin{flalign}
i\hbar \frac{\partial}{\partial t} c_k\left(t\right)
& = \sum_n c_n\left(t\right) e^{-\frac{i}{\hbar}\left(E_n - E_k\right) t} 
\int \phi_k^{(0)*} \hat{H}_I\left(t\right) \phi^{(0)}_n d\vec{x}
\\
& = e^{-\frac{i}{\hbar}\left(E_1 - E_k\right) t} 
\int \phi_k^{(0)*} \left[\hat{H}_I e^{-\frac{iEt}{\hbar}}\right] \phi^{(0)}_1 d\vec{x}
\\
& = e^{-i\left(\omega_1 - \omega_k\right) t} 
\int \phi_k^{(0)*} \left[\hat{H}_I e^{-i\omega t}\right] \phi^{(0)}_1 d\vec{x}
\\
& = e^{-i\left(\omega_1 - \omega_k\right) t} e^{-i\omega t}
\int \phi_k^{(0)*} \hat{H}_I \phi^{(0)}_1 d\vec{x}
\end{flalign}
We define $\Delta\omega = \omega - (\omega_k - \omega_1)$ as the \textit{detuning}, the frequency
mismatch between the electromagnetic wave and the frequency associated with the energy difference
between $\phi^{(0)}_1$ and $\phi^{(0)}_k$, and simplify:
\begin{flalign}
i\hbar \frac{\partial}{\partial t} c_k\left(t\right)
& = e^{-i\Delta\omega t} \int \phi_k^{(0)*} \hat{H}_I \phi^{(0)}_1 d\vec{x}
\end{flalign}
Now that we have a simple expression for the change in $c_n\left(t\right)$, we can just integrate
this to get the probability amplitude of finding the atom in state $\psi_k^{(0)}$ at time $t$.
\begin{flalign}
c_n\left(t\right) = \int_{0}^{t} \frac{\partial}{\partial t} c_k\left(t'\right) dt'
& = \frac{1}{i\hbar} \int_{0}^{t} e^{-i\Delta\omega t'} 
\int \phi_k^{(0)*} \hat{H}_I \phi^{(0)}_1 d\vec{x} \, dt'
\\
& = \frac{1}{i\hbar} \int_{0}^{t} e^{-i\Delta\omega t'} 
\langle \phi_k^{(0)} \vert \hat{H}_I \vert \phi^{(0)}_1 \rangle dt'
\end{flalign}
This looks similar to a Fourier transform, but instead of integrating over all time, the integral
only takes place between times $0$ and $t$. This is equivalent to multiplying by the boxcar function
defined such that it is equal to 1 inside the interval and 0 outside of it. We can also pull out the
$\langle \phi_k^{(0)} \vert \hat{H}_I \vert \phi^{(0)}_1 \rangle$ term, as we have removed the time
dependence from $\hat{H}_I$:
\begin{flalign}
c_n\left(t\right) 
& = \frac{1}{i\hbar} \langle \phi_k^{(0)} \vert \hat{H}_I \vert \phi^{(0)}_1 \rangle
\int_{-\infty}^{\infty} \text{boxcar}\left(0, t\right) e^{-i\Delta\omega t'}  dt'
\end{flalign}
The Fourier transform of the boxcar function is a sinc function multiplied by a shift factor:
\begin{flalign}
c_n\left(t\right) 
& = \frac{1}{i\hbar} \langle \phi_k^{(0)} \vert \hat{H}_I \vert \phi^{(0)}_1 \rangle
e^{-\pi i \Delta\omega t} \frac{\sin\left(\frac{1}{2}\Delta\omega t\right)}{\frac{1}{2}\Delta\omega}
\end{flalign}
To get the probability, we use the modulus as per the Born rule:
\begin{flalign}
\lvert c_n\left(t\right) \rvert^2 = c_n\left(t\right)^* c_n\left(t\right)
& = \frac{1}{\hbar^2} \vert\langle \phi_k^{(0)} \vert \hat{H}_I \vert \phi^{(0)}_1 \rangle\vert^2
    \left(\frac{\sin\left(\frac{1}{2}\Delta\omega t\right)}{\frac{1}{2}\Delta\omega}\right)^2
\end{flalign}

The Fourier transform allows us to make a statement about the pulse size and the frequency mismatch.
An increasing pulse duration will decrease the width of its Fourier transform, meaning that very
short pulses have a higher probability of exciting states far from the frequency of the
electromagnetic wave. This is an expression of the uncertainty principle, which holds not just for
quantum systems, but any system that admits wavelike behavior. With a longer pulse, the mismatch
tolerance is reduced, but we are more likely to only excite the system to a state whose energy
difference matches that frequency.

\subsection{A continuum of states}

A real system will likely consist of many state that are accessible by a short excitation. If we
have a near-continuum of possible final states, we may want to determine the probability of the
final state being within the continuum.

The number of states $N$ within some energy interval $[E_{\text{min}}, E_{\text{max}}]$ is given
by the integral of the density of states (DOS) function:
\begin{equation}
N = \int_{E_{\text{min}}}^{E_{\text{max}}} \rho\left(E\right) dE
\end{equation}

